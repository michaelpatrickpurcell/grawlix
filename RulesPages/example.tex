\setmainfont{Quicksand}[
	UprightFont = *-Medium,
	BoldFont = *-Bold]
\raggedright

\section{Example}
Seventeen tiles are in the grid. The next player can place their tile in one of sixteen possible locations. These locations are indicated by the dotted squares in the diagram below.

The grid has six columns, so a tile cannot be placed in a new column. The grid only has five rows, so a tile can be placed in a new row.

\begin{figure}[h]
\centering
\begin{tikzpicture}[transform shape, scale=2.5]

\node[vacant] at (0,0.5) {};
\node[dotter] at (0,0.5) {\Large{?}};
\node[vacant] at (0.5,0.5) {};
\node[dotter, rotate=180] at (0.5,0.5) {};
\node[vacant] at (2.5,0.5) {};
\node[dotter, rotate=180] at (2.5,0.5) {};

\node[vacant] at (1,0) {};
\node[dotter, rotate=180] at (1,0) {};
\node[vacant] at (2,0) {};
\node[dotter] at (2,0) {};
\node[vacant] at (1.5,-0.5) {};
\node[dotter] at (1.5,-0.5) {};
\node[vacant] at (2,-0.5) {};
\node[dotter] at (2,-0.5) {};
\node[vacant] at (1.5,-1) {};
\node[dotter] at (1.5,-1) {};

\node[vacant] at (0,-1) {};
\node[dotter] at (0,-1) {};
\node[vacant] at (1,-1) {};
\node[dotter] at (1,-1) {};

\node[vacant] at (0,-2) {};
\node[dotter] at (0,-2) {};
\node[vacant] at (0.5,-2) {};
\node[dotter] at (0.5,-2) {};
\node[vacant] at (1.5,-2) {};
\node[dotter] at (1.5,-2) {};
\node[vacant] at (2,-2) {};
\node[dotter] at (2,-2) {};
\node[vacant] at (2.5,-2) {};
\node[dotter] at (2.5,-2) {};

\node[vacant] at (1,-2.5) {};
\node[dotter] at (1,-2.5) {\Large{!}};


\foreach \i/\j/\k in {0.0/blue/at, 0.5/violet/ampersand, 2.5/orange/percent} {
	\node[tile] () at (\i,0.0) {};
	\pic () at (\i,0.0) {\k={\j}};
}

\foreach \i/\j/\k in {0.0/orange/asterisk, 0.5/red/dollar, 1.0/violet/at, 2.5/green/octothorpe} {
	\node[tile] () at (\i,-0.5) {};
	\pic () at (\i,-0.5) {\k={\j}};
}

\foreach \i/\j/\k in {0.5/blue/octothorpe, 2.0/green/at, 2.5/yellow/ampersand} {
	\node[tile] () at (\i,-1.0) {};
	\pic () at (\i,-1.0) {\k={\j}};
}

\foreach \i/\j/\k in {0.0/green/dollar, 0.5/yellow/asterisk, 1.0/blue/percent, 1.5/violet/octothorpe, 2.0/orange/ampersand, 2.5/red/at} {
	\node[tile] () at (\i,-1.5) {};
	\pic () at (\i,-1.5) {\k={\j}};
}

\foreach \i/\j/\k in {1.0/orange/octothorpe} {
	\node[tile] () at (\i,-2.0) {};
	\pic () at (\i,-2.0) {\k={\j}};
}

\end{tikzpicture}

\end{figure}

Consider the location labelled with the question mark. There are no tiles in this row. The tiles in this column are the blue \smallat, orange \smallasterisk, and green \smalldollar. So, the allowed glyphs are: \smallpound, \smallpercent, \smallampersand. The allowed colors are: red, yellow, violet.

If a tile is placed in the location labelled with the question mark, then the grid will have six rows. If that happens, then future tiles cannot be placed in a new row. In particular, the location labelled with the exclamation point will become unavailable.

